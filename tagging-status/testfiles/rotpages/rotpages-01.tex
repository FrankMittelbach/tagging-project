\DocumentMetadata
  {
    lang=en-US,
    pdfversion=2.0,
    pdfstandard=ua-2,
    testphase={phase-III,title,math,table,firstaid}
  }
\documentclass[12pt,twoside]{article}
\usepackage[a4paper,headheight=4ex]{geometry}
\usepackage{rotpages}
\usepackage{fancyhdr}

\title{\texttt{rotpages.sty} --- Multiple page rotation in \LaTeX\\
  Example file}
\author{Sergio Callegari}
\date{January 1st, 2003}

\fancyhead[LE]{\emph{rotpages.sty}}
\fancyhead[RO]{Multiple page rotation in \LaTeX}
\fancyfoot{}
\fancyfoot[L]{{\footnotesize Example file for a single column document.}}
\fancyfoot[R]{\thepage}


\begin{document}
\sloppy
\pagestyle{fancy}
\maketitle
\thispagestyle{fancy}

This is the first page of the document. In this document we take
advantage only of the simplest features of the \texttt{rotpages.sty}
package, i.e.\@ we only use the basic \verb|\rotboxpages| and
\verb|\endrotboxpages| commands.


The first pages are typeset normally.  To fill them a little, we include
the first part of \emph{Pinocchio} by Carlo Collodi.

\bigskip
{\slshape%
  \textbf{CHAPTER 1}
  
  \emph{How it happened that Mastro Cherry, carpenter,
  found a piece of wood that wept and laughed like a child}
  \medskip
 
  Centuries ago there lived ---
  
  "A king!" my little readers will say immediately.
  
  No, children, you are mistaken.  Once upon a time
  there was a piece of wood.  It was not an expensive piece
  of wood.  Far from it.  Just a common block of firewood,
  one of those thick, solid logs that are put on the fire in
  winter to make cold rooms cozy and warm.
  
  I do not know how this really happened, yet the fact
  remains that one fine day this piece of wood found itself
  in the shop of an old carpenter.  His real name was
  Mastro Antonio, but everyone called him Mastro Cherry,
  for the tip of his nose was so round and red and shiny
  that it looked like a ripe cherry.
  
  As soon as he saw that piece of wood, Mastro Cherry
  was filled with joy.  Rubbing his hands together happily,
  he mumbled half to himself:
  
  "This has come in the nick of time.  I shall use it to
  make the leg of a table."
  
  He grasped the hatchet quickly to peel off the bark and
  shape the wood.  But as he was about to give it the first
  blow, he stood still with arm uplifted, for he had heard a
  wee, little voice say in a beseeching tone:  "Please be careful!
  Do not hit me so hard!"
  
  What a look of surprise shone on Mastro Cherry's
  face!  His funny face became still funnier.
  
  He turned frightened eyes about the room to find out
  where that wee, little voice had come from and he saw
  no one! He looked under the bench--no one! He peeped
  inside the closet--no one! He searched among the shavings--
  no one! He opened the door to look up and down
  the street--and still no one!

  "Oh, I see!" he then said, laughing and scratching his Wig.
  "It can easily be seen that I only thought I heard the tiny
  voice say the words! Well, well--to work once more."
  
  He struck a most solemn blow upon the piece of wood.
  
  "Oh, oh!  You hurt!" cried the same far-away little voice.
  
  Mastro Cherry grew dumb, his eyes popped out of his
  head, his mouth opened wide, and his tongue hung down
  on his chin.
  
  As soon as he regained the use of his senses, he said,
  trembling and stuttering from fright:
  
  "Where did that voice come from, when there is no
  one around?  Might it be that this piece of wood has
  learned to weep and cry like a child?  I can hardly
  believe it.  Here it is--a piece of common firewood, good
  only to burn in the stove, the same as any other.  Yet--
  might someone be hidden in it?  If so, the worse for him.
  I'll fix him!"
  
  With these words, he grabbed the log with both hands
  and started to knock it about unmercifully.  He threw it
  to the floor, against the walls of the room, and even up
  to the ceiling.
  
  He listened for the tiny voice to moan and cry.
  He waited two minutes--nothing; five minutes--nothing;
  ten minutes--nothing.
  
  "Oh, I see," he said, trying bravely to laugh and
  ruffling up his wig with his hand.  "It can easily be seen
  I only imagined I heard the tiny voice!  Well, well--to
  work once more!"

  The poor fellow was scared half to death, so he tried
  to sing a gay song in order to gain courage.
  
  He set aside the hatchet and picked up the plane to
  make the wood smooth and even, but as he drew it to
  and fro, he heard the same tiny voice.  This time it giggled
  as it spoke:
  
  "Stop it!  Oh, stop it!  Ha, ha, ha! You tickle my stomach."
  
  This time poor Mastro Cherry fell as if shot.  When
  he opened his eyes, he found himself sitting on the floor.
  
  His face had changed; fright had turned even the tip of
  his nose from red to deepest purple.

}
\bigskip
Note that the next pages are upside down.

\rotboxpages%
Here come the rotated pages. Note that while formatting the
document, this page is \emph{deferred}, until all the block of rotated
pages is processed. In this way, this page is printed as the last one
of the block. However, if the printed work is read upside down, this
page correctly appears as the first of the block.

Obviously, also this page contains the continuation of the novel:

\bigskip
{\slshape%

  \textbf{CHAPTER 2}

  \emph{Mastro Cherry gives the piece of wood to his friend Geppetto,
    who takes it to make himself a Marionette that will dance,
    fence, and turn somersaults}
  \medskip
  
  In that very instant, a loud knock sounded on the door.
  "Come in," said the carpenter, not having an atom of
  strength left with which to stand up.
  
  At the words, the door opened and a dapper little old man came in.
  His name was Geppetto, but to the boys of the neighborhood he was
  Polendina\footnote{Corneal mush}, on account of the wig he always
  wore which was just the color of yellow corn.
  
}

\bigbreak

Here we make a small break in the story. Please, take a second to
observe how the page content is rotated, while the page headers and
footers, comprising the page number, are printed with the standard
orientation. Take also a quick look at the footnote and observe that
it is in the right place.

Note also that in order to introduce a frame, the rotated pages are
slightly smaller (i.e.\@ they contain a little less text than the
normal ones.)  

After this informative bit, it is time for some more Pinocchio:

\bigskip
{\slshape%
  Geppetto had a very bad temper.  Woe to the one who
  called him Polendina!  He became as wild as a beast and
  no one could soothe him.
  
  "Good day, Mastro Antonio," said Geppetto.  "What
  are you doing on the floor?"
  
  "I am teaching the ants their A B C's."
  
  "Good luck to you!"
  
  "What brought you here, friend Geppetto?"
  
  "My legs.  And it may flatter you to know, Mastro
  Antonio, that I have come to you to beg for a favor."
  
  "Here I am, at your service," answered the carpenter,
  raising himself on to his knees.
  
  "This morning a fine idea came to me."
  
  "Let's hear it."
  
  "I thought of making myself a beautiful wooden
  Marionette.  It must be wonderful, one that will be able to
  dance, fence, and turn somersaults.  With it I intend to go
  around the world, to earn my crust of bread and cup of
  wine.  What do you think of it?"
  
  "Bravo, Polendina!" cried the same tiny voice which
  came from no one knew where.
  
  On hearing himself called Polendina, Mastro Geppetto
  turned the color of a red pepper and, facing the carpenter,
  said to him angrily:
  
  "Why do you insult me?"
  
  "Who is insulting you?"
  
  "You called me Polendina."
  
  "I did not."
  
  "I suppose you think \_I\_ did!  Yet I KNOW it was you."
  
  "No!"
  
  "Yes!"
  
  "No!"
  
  "Yes!"
  
  And growing angrier each moment, they went from
  words to blows, and finally began to scratch and bite and
  slap each other.
  
  When the fight was over, Mastro Antonio had Geppetto's
  yellow wig in his hands and Geppetto found the carpenter's
  curly wig in his mouth.
  
  "Give me back my wig!" shouted Mastro Antonio in a surly voice.
  
  "You return mine and we'll be friends."
  
  The two little old men, each with his own wig back on
  his own head, shook hands and swore to be good friends
  for the rest of their lives.
  
  "Well then, Mastro Geppetto," said the carpenter, to
  show he bore him no ill will, "what is it you want?"
  
  "I want a piece of wood to make a Marionette.  Will you give it to
  me?"
    
  Mastro Antonio, very glad indeed, went immediately
  to his bench to get the piece of wood which had frightened
  him so much.  But as he was about to give it to his friend,
  with a violent jerk it slipped out of his hands and hit
  against poor Geppetto's thin legs.
  
  "Ah!  Is this the gentle way, Mastro Antonio, in which
  you make your gifts?  You have made me almost lame!"
  
  "I swear to you I did not do it!"
  
  "It was \_I\_, of course!"
  
  "It's the fault of this piece of wood."
  
  "You're right; but remember you were the one to throw it at my legs."
  
  "I did not throw it!"
  
  "Liar!"
  
  "Geppetto, do not insult me or I shall call you Polendina."
  
  "Idiot."
  
  "Polendina!"
  
  "Donkey!"
  
  "Polendina!"
  
  "Ugly monkey!"
  
  "Polendina!"
  
  On hearing himself called Polendina for the third time,
  Geppetto lost his head with rage and threw himself upon
  the carpenter.  Then and there they gave each other a
  sound thrashing.
  
  After this fight, Mastro Antonio had two more scratches
  on his nose, and Geppetto had two buttons missing from
  his coat.  Thus having settled their accounts, they shook
  hands and swore to be good friends for the rest of their lives.
  
  Then Geppetto took the fine piece of wood,
  thanked Mastro Antonio, and limped away toward home.
}
\endrotboxpages 

And the normal behaviur of \LaTeX is back!
Exiting isn't it? So don't forget to tell your friends about this new
package!
\end{document}