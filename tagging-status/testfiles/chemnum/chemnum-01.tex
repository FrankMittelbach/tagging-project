\DocumentMetadata
  {
    lang=en-US,
    pdfversion=2.0,
    pdfstandard=ua-2,
    testphase={phase-III,title,math,table,firstaid}
  }
\documentclass{article}

\usepackage{chemnum}
\usepackage{hyperref}

\setchemnum{hyperlinks}

\title{chemnum tagging test}

\begin{document}

Compounds \cmpd{a} and \cmpd{b} are declared and can be used any time:
\cmpd{a}. No pre-declaring is necessary. Compounds like \cmpd{c} are
numbered in the order they appear in the text.\par
Once again: \cmpd{b}, \cmpd{a}, \cmpd{c}.

The hidden version\cmpd*{d} declares the label but doesn't print anything.
The next \cmpd{e} continues to count with the next number. With \cmpd{d}
the label can be used, of course.

\cmpd{f.one} and \cmpd{f.two} are related, as are \cmpd{g.one} and
\cmpd{g.two}. Of course these labels can be used again: \cmpd{g.two} and
\cmpd{f.one}.

\cmpd{a} and its variants \cmpd{a.one} and \cmpd{a.two}

list of labels: \cmpd{q.one, q.two, q.three, q.four, q.five}\par
label with list of sublabels: \cmpd{q.{one,two,three,four,five}}

\setchemnum{compress=false}%
list of labels: \cmpd{q.one, q.two, q.three, q.four, q.five}\par
label with list of sublabels: \cmpd{q.{one,two,three,four,five}}

\end{document}
