\DocumentMetadata{
   lang=en,
   pdfversion=2.0,
   pdfstandard=ua-2,
   pdfstandard=a-4f, %  AF not in document catalog dictionary
   testphase = {phase-III,math,title}}


\documentclass[12pt]{article}

% ----------------------------------------------------------------------
% General packages
% ----------------------------------------------------------------------
\usepackage[UKenglish]{babel}
\usepackage{unicode-math}
\usepackage[final]{microtype}

\usepackage{csquotes}
\usepackage{geometry}
\usepackage{siunitx}
\usepackage{hyperref}
% ----------------------------------------------------------------------

% ----------------------------------------------------------------------
% Set up
% ----------------------------------------------------------------------
\geometry{margin = 1in}
\hypersetup{hidelinks,unicode}
% ----------------------------------------------------------------------

% ----------------------------------------------------------------------
% New commands for this document
% ----------------------------------------------------------------------
\NewDocumentCommand\acro{m}{\textsc{#1}}
\NewDocumentCommand\email{m}{\href{mailto:#1}{#1}}
\NewDocumentCommand\foreign{m}{\emph{#1}}

% ----------------------------------------------------------------------

% ----------------------------------------------------------------------
% Meta-data
% ----------------------------------------------------------------------
\title{Group Theory and Symmetry}
\author{Dr Joseph Wright\\\email{joseph.wright@uea.ac.uk}}
\date{PHY-5250A\\2023--24}
% ----------------------------------------------------------------------

\begin{document}

\maketitle

\section{Group axioms}

\begin{enumerate}
  \item Multiplication: there is a combination rule for elements of the set,
    generally denoted as~$\circ$, as in $A \circ B$
  \item Closure: multiplication always gives another member of the set
  \item Identity: there is an element of the set~($I$) that leaves other elements
    unchanged on multiplication, $A \circ I = A$
  \item Inverse: For each element of the set, there is a element we can combine
    it with that will give us the identity, $A \circ A^{-1} = I$
  \item Associativity: for three elements of the set $A$, $B$ and $C$,
    $A \circ (B \circ C) = (A \circ B) \circ  C$
\end{enumerate}

For an \emph{Abelian} group, we also have commutativity, $A \circ B = B \circ A$.

Two groups are \emph{isomorphic} if we can \enquote{match up} the elements:
they have the same pattern in their multiplication table for example.

\section{Symmetry elements}

\begin{description}
  \item[$E$] Identity
  \item[$C_{n}$] $n$-Fold rotation, \foreign{i.e.}~through
    \ang[parse-numbers = false]{(360/n)}
  \item[$\sigma$] Mirror plan
  \item[$i$] Centre of inversion
  \item[$S_{n}$] Improper rotation, \foreign{i.e.}~rotation through
    \ang[parse-numbers = false]{(360/n)} then reflection
    perpendicular to the axis
\end{description}

Formally, symmetry \emph{operations} need the \enquote{hat} symbol, for example
$\hat{C}_{4}$ is the symmetry operation for a rotation of \ang{90}: these are
rarely seen in the molecular symmetry literature, so we choose to omit them.

\subsection{Reducible and irreducible representations}

The symbol $\Gamma$ is the representation of a point group which can either be
reducible or written as a linear combination of irreducible representations.

The \emph{projection formula} allows you to do this process simply by testing
each line of the character table to see if it contributes to the
representation~$\chi$. It is
\begin{equation}
  a_{i} = \frac{1}{h}\sum_{\mathrm{C}}\chi(R)\chi_{i}(R)g(\mathrm{C})
  \label{eq:projection}
\end{equation}
where
\begin{description}
  \item[$a_{i}$] is the number of times that the $i$th line of the character
    table contributes;
  \item[$h$] is the order of the point group (the total number of unique
    operations);
  \item[$\chi(R)$] is the $R$th member of the representation being
    decomposed;
  \item[$\chi_{i}(R)$] is the $R$th member of line $i$ in the character
    table
  \item[$g(\mathrm{C})$] is the number of elements in the class
    of symmetry operation.
\end{description}

\section{Allowed transitions}

In general, a transition is allowed when
\[
 \Gamma(\Psi_{\mathrm{i}}) \times
 \Gamma(\mu) \times
 \Gamma(\Psi_{\mathrm{f}}) \supset A
\]

\section{Vibrational spectroscopy}

\begin{enumerate}
  \item Form a reducible representation using $x$, $y$, and $z$ matrices
    on each atom
  \item Express this representation in terms of irreducible representations
    using Equation~\ref{eq:projection}
  \item Discard irreducible representations corresponding to rotations
    and translations
  \item IR active vibrations allowed based on dipole ($x$, $y$, $z$ components);
    Raman active vibrations allowed based on quadropole (tensor terms)
\end{enumerate}


\end{document}
